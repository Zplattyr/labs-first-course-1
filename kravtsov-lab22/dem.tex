\documentclass[12pt]{book}
\usepackage{graphicx}
\usepackage[russian]{babel}
\usepackage{amsmath}
\usepackage{ amssymb }
\usepackage{ textcomp }
\usepackage{ esint }
\usepackage{geometry}
\usepackage{ upgreek }
\geometry{papersize={15.3 cm,29 cm}}
\geometry{left=0.68cm}
\geometry{right=0.68cm}
\geometry{top=1.4cm}
\geometry{bottom=0.4cm}
\setcounter{page}{6}

\usepackage{fancyhdr}

\begin{document}

\begin{center}
\large\textsc{{РАЗДЕЛ I}}
\vspace{0.5cm} 

\Large{\textbf{ВВЕДЕНИЕ В АНАЛИЗ}}

\vspace{4cm} 

\textbf{\large{$\S$ 1. Вещественные числа}}
\end{center}

\textbf{1. Метод математической индукции.} Чтобы доказать, что некоторая теорема верна для всякого натурального числа $n$, достаточно доказать: 1) что эта теорема справедлива для $n=1$ и 2) что если эта теорема справедлива для какого-нибудь натурального числа $n$, то она справедлива также и для следующего натурального числа $n+1$.

\textbf{2. Сечение.} Разбиение рациональных чисел на два класса $A$ и $B$ называется сечением, если выполнены следующие условия: 1) оба класса не пусты; 2) каждое рациональное число попадает в один и только в один класс и 3) любое число, принадлежащее классу $A$ (нижний класс), меньше произвольного числа, принадлежащего классу $B$ (верхний класc). Ceчение $A / B$ определяет: а) рациональное число, если или нижний класс $A$ имеет наибольшее число или же верхний класс $B$ имеет наименьшее число, и б) иррациональное число, если класс $A$ не имеет наибольшего числа, а класс $B$ - наименьшего числа. Числа рациональные и иррациональные носят название вещественных или действительныx\footnote{$^{)}$ В дальнейшем под словом число мы будем понимать вещественное число, если не оговорено противное.}$^{)}$.

\textbf{3. Абсолютная величина (или модуль).} Если $x$ - вещественное число, то абсолютной величиной (модулем) $|x|$ называется неотрицательное число, определяемое следующими условиями:

$$
|x|=\left\{\begin{aligned}
-x, & \text { если } x<0; \\
x, & \text { если } x \geqslant 0 .
\end{aligned}\right.
$$

Для любых вещественных чисел $x$ и $y$ имеют место неравенства
$$
|x|-|y| \leqslant|x+y| \leqslant|x|+|y| \text {. }
$$

\textbf{4. Верхняя и нижняя грани.} Пусть $X=\{x\}$ - ограниченное множество вещественных чисел. Число
$$
m=\inf \{x\}
$$ 
называется нижней гранью множества $X$, если:

1) каждое $x \in X$\footnote{$^{)}$ Запись $x \in X$ означает, что число $x$ принадлежит множеству $X$.}$^{)}$
 удовлетворяет неравенству
$$
x \geqslant m
$$

2) каково бы ни было $\varepsilon>0$, существует $x^{\prime} \in X$ такое, что
$$
x^{\prime}<m+\varepsilon .
$$

\pagestyle{fancy}
\fancyhead{}
\fancyhead[CO]{\& 1. Вещественные числа}
\fancyhead[RO]{\thepage}

Аналогично число
$$
M=\sup \{x\}
$$
называется верхней гранью множества $X$, если:

1) каждое $x \in X$ удовлетворяет неравенству
$$
x \leqslant M,
$$

2) для любого $\varepsilon>0$ существует $x^{\prime \prime} \in X$ такое, что
$$
x^{\prime \prime}>M-\varepsilon \text {. }
$$

Если множество $X$ не ограничено снизу, то принято говорить, что
$$
\inf \{x\}=-\infty
$$
если же множество $X$ не ограничено сверху, то полагают
$$
\sup \{x\}=+\infty
$$

\textbf{5. Абсолютная и относительная погрешности.} Если $a(a \neq 0)$ есть точное значение измеряемой величины, а $x$ - приближенное значение этой величины, то
$$
\Delta=|x-a|
$$
называется абсолютной погрешностью, а
$$
\delta=\frac{\Delta}{|a|}
$$
- относительной погрешностью измеряемой величины.

Говорят, что число $x$ имеет $n$ верных знаков, если абсолютная погрешность этого числа не превышает половины единицы разряда, выражаемого $n$-й значащей цифрой.

Применяя метод математической индукции, доказать, что для любого натурального числа $n$ справедливы следующие равенства:

\textbf{1.} $1+2+\ldots+n=\frac{n(n+1)}{2}$.

\textbf{2.} $1^2+2^2+\ldots+n^2=\frac{n(n+1)(2 n+1)}{6}$.

\textbf{3.} $1^3+2^3+\ldots+n^3=(1+2+\ldots+n)^2$.

\textbf{4.} $1+2+2^2+\ldots+2^{n-1}=2^n-1$.

\textbf{5.} Пусть
$$
a^{[n]}=a(a-h) \ldots[a-(n-1) h] \text { и } a^{[0]}=1 .
$$

Доказать, что

$$
(a+b)^{[n]}=\sum_{m=0}^n C_n^m a^{[n-m]} b^{[m]},
$$
где $C_n^m$ - число сочетаний из $n$ элементов по $m$. Вывести отсюда формулу бинома Ньютона.

\end{document}